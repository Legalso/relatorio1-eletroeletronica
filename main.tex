\documentclass[a4paper]{article}
\usepackage[utf8]{inputenc}
\usepackage{graphicx}
\usepackage{amsmath}
\usepackage[bottom=2.0cm,top=2.0cm,left=2.0cm,right=2.0cm]{geometry}
\usepackage[portuges]{babel}
\usepackage{indentfirst}
\usepackage{hyperref}  
\hypersetup{colorlinks,citecolor=black,filecolor=black,linkcolor=black,urlcolor=black}
\usepackage[nottoc]{tocbibind}
\usepackage{lipsum}
\usepackage{scalefnt}
\usepackage{array}
\usepackage{booktabs}
\usepackage{siunitx}
\usepackage{xcolor}

\usepackage{tocloft}

\begin{document}

% Capa
\begin{titlepage}
    \begin{center}
        \includegraphics[width=3.9cm]{imagens/LogoSENAC.png}\\[1cm]
        \Large{\textbf{CENTRO UNIVERSITÁRIO SENAC}}\\[4cm]
        
        \LARGE{\textbf{RELATÓRIO DE ELETROELETRÔNICA}}\\[0.5cm]
        \Large{EXPERIÊNCIA 1}\\[3cm]
        
        \large
        CAIO PATRIOTA BARROS \\
        LUIZ FELIPE PEREIRA ARAÚJO \\
        MATHEUS EMANUEL RODRIGUES ARAÚJO \\
        ANA GIULIA DE ALMEIDA ROSSINI \\
        THIAGO RIVAS CABALLERO \\
        
        \vfill
        São Paulo \\
        \today
    \end{center}
\end{titlepage}

% Sumário
\newpage
\renewcommand{\contentsname}{Sumário}

\renewcommand{\cftsecleader}
{\cftdotfill{\cftdotsep}}

\tableofcontents
\newpage

\pagenumbering{arabic}
\large

\section{Resumo}
De forma breve, descreve qual é o objetivo do mesmo, o que foi feito e quais foram os resultados obtidos. Embora apareça no início do relatório, usualmente é a última parte a ser escrita.

\section{Introdução}
O relatório deve apresentar os resultados obtidos de forma clara e concisa. É necessário expor cuidadosamente quais são os objetivos do trabalho realizado, os conceitos físicos necessários para a realização do experimento e como o experimento foi realizado, entre outros. O relatório tem que ser escrito de modo que um leitor seja capaz de entender e até reproduzir o trabalho a partir do conhecimento adquirido na sua leitura.

Nela expõem-se as motivações do trabalho e os objetivos a serem atingidos. Apresenta uma revisão das informações existentes sobre o tema. Deve incluir uma explicação teórica mínima (não copiada de livro, mas elaborada pelos alunos) que permita a compreensão do trabalho.

\begin{equation}
    \Delta \textbf{s} = \Delta \vec{s}
\end{equation}

\section{Objetivo}
O objetivo...

\section{Materiais}
Os materiais...

\section{Metodologia}
Deve-se descrever em detalhe a configuração experimental utilizada, os métodos empregados para a realização das medidas, incluindo a fundamentação física.

Deve conter uma descrição de aspectos relevantes dos dispositivos e equipamentos utilizados, especificando suas características importantes (precisão dos instrumentos, intervalos de medição, etc). Pode-se representar esquematicamente o dispositivo empregado para a realização do experimento.

\section{Resultados}
Esta seção é a continuação natural da Introdução e do Método Experimental. Deve incluir tabelas dos dados obtidos junto com as suas incertezas e a explicação de como foram avaliadas essas incertezas. Também deve incluir a descrição de como a análise de dados foi realizada e como os resultados foram obtidos. Deverá incluir gráficos e tabelas junto com as curvas de ajuste dos dados realizados.

\begin{table}[h]
    \centering
    \scalefont{1}
    \begin{tabular}{cccc}
        \hline
        Símbolo & Prefixo & Múltiplo \\
        \hline \hline
        E & exa & $10^{18}$ \\
        P & peta & $10^{15}$ \\
        T & tera & $10^{12}$ \\
        G & giga & $10^{9}$ \\
        M & mega & $10^{6}$ \\
        k & kilo & $10^{3}$ \\
        h & hecto & $10^{2}$  \\
        da & deca & $10^{1}$ \\
        d & deci & $10^{-1}$ \\
        c & centi & $10^{-2}$ \\
        m & mili & $10^{-3}$ \\
        $\mu$ & micro & $10^{-6}$ \\
        n & nano & $10^{-9}$ \\
        p & pico & $10^{-12}$ \\
        f & femto & $10^{-15}$ \\
        a & atto & $10^{-18}$ \\
        \hline
    \end{tabular}
    \caption{Prefixos das potências de 10}
    \label{tbl:prefixos}
\end{table}

\section{Conclusões}
Deve conter uma discussão dos resultados obtidos à luz das hipóteses e objetivos do trabalho. A discussão do trabalho deverá ser feita de forma crítica, propondo melhorias ao trabalho realizado, tanto na metodologia empregada quanto nas propostas para ampliar o objetivo do experimento no futuro.

\section{Referências}
CRUZ, C. C. A.; MARKUS, O. Circuitos Elétricos – Corrente Contínua e Corrente Alternada. [s.l.] Saraiva Educação S.A., 2018.

\end{document}
